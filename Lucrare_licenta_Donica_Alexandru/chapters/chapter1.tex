

\chapter{Algoritmul DPLL (Davis-Putnam-Logemann-Loveland)}

Algoritmul DPLL pentru rezolvarea problemei SAT este considerat o optimizare a metodei de backtracking și stă la baza creării a multor 'SAT solveri' moderni.

În cadrul acestui algoritm a fost demonstrată proprietatea ca pentru orice formulă care se poate scrie în formă normală conjunctivă (CNF), se va returna o soluție validă dacă aceasta există, și se va specifica mereu când nu există dacă formula este într-adevăr nesatisfiabilă.

Optimizarea cea mai importantă pe care acest algoritm o aduce metodei \linebreak backtracking este 'propagarea clauzelor unit', unde o 'clauză unit' este o clauză ce nu conține nici un literal cu valoarea \textit{ True } și care conține un singur literal neasignat încă. 

Astfel, la fiecare pas recursiv, se caută fiecare clauză 'unit' și se asignează o \linebreak valoare variabilei rămase, astfel încât literalul din clauza respectivă să aibe \linebreak valoarea \textit{ True }. Motivul pentru care această metodă funcționează este că dacă s-ar asigna cealaltă valoare de adevăr unui astfel de literal, valoarea de adevăr a clauzei respective ar fi \textit{False}, deci soluția ar fi invalidă.

Urmează pseudo-codul metodei recursive '\textit{dpll}', formulată după algoritmul DPLL modificat sa corespundă implementării acestui proiect.


\iftrue
\begin{algorithm}
	\KwIn{formula : f \{ length f $\ge$ 1 \} ; \linebreak truth\_assignment : tau\{ length tau $\leq$ length f $\land$ is\_partial\_solution f tau \}}
	\KwOut{ result : r \{ (r is (Sat tau) $\implies$ \textit{calculate\_formula\_value} f tau =  True  ) 
	\newline $\land$ (r is Unsat $\implies$ $\forall$ truth\_assignment : t . \newline \textit{calculate\_formula\_value} f ta = false) \} } 
	
	
	\If{\textit{length tau = length (\textbf{variables\_of\_f} f)}}
	{
	\Return{Sat tau}
	}
	\Else{
	
	\textit{new\_var} $\gets$ \textit{\textbf{get\_unassigned\_variable } f tau} \newline
	\textit{tau\_2} $\gets$\textit{ tau[ new\_var : False]} 	\newline
	\textit{tau\_2 , new\_literals} $\gets$ \textit{\textbf{unit\_clause\_propagation} f tau\_2} 	\newline
	\textit{clauses\_with\_new\_literals} $\gets$ \textit{get\_clauses\_that\_contains\_min\_1\_literal f new\_literals}  \newline
	\If{
	$\exists$ c : clause {c $\in$ \textit{clauses\_with\_new\_literals} 
	$\land$ is\_clause\_false c tau\_2 }
	}
	{
		\textit{tau\_2} $\gets$\textit{ tau[ new\_var : True ]}
	 \newline	...aceeiasi pasi ca pana acum, dar cu noul \textit{tau\_2} ,și returneaza \textit{NotSat} în block-ul \textit{if-then} corespondent...
	}
	\Else{
		\textit{res} $\gets$ \textit{dpll\_recursiv f tau\_2} \newline
		\If{\textit{res is NotSat}}
		{
			\textit{tau\_2} $\gets$\textit{ tau[ new\_var : True ]}
			 \newline	...aceeiasi pasi ca pana acum, dar cu noul \textit{tau\_2} ,si returneaza \textit{NotSat} în block-ul \textit{if-then} corespondent...
		}
		\Else{\Return{res}}
	}
	}
	\caption{dpll-recursiv}
\end{algorithm}
\fi
\newpage


Exemplu: formulă ce conține 5 clauze, numerotate de la $c_1$ la $c_5$ și 6 variabile numerotate de la $ x_1 $ la $x_6$

( $\neg x_1 \lor x_3 \lor x_4 $) $\land$
($\neg x_2 \lor \neg x_3 \lor x_5$) $\land$
($x_1 \lor \neg x_2$) $\land$
($x_4  \lor \neg x_5$) $\land$
($x_1 \lor \neg x_4 \lor \neg x_6$)

Implementarea acestui proiect ar începe căutarea unei soluții asignând unei variabile noi o valoare oarecare, mai exact, variabilei $x_1$ valoarea \textit{False}. Urmează etapa de unit propagation, în care se găsește clauza $c_3$ a fi clauză unitară, și se asignează \linebreak variabilei $x_2$ valoarea \textit{False}.
Apoi, recursiv, se ia logica de la capăt cu datele noi obținute. Se va găsi primul literal neasignat $x_3$ și i se va acorda valoarea \textit{False}. În etapa de propagare a clauzelor unitare se găsește prima clauză ce conține 2 literali cu valoare falsă și variabila $x_4$, care va primi valoarea \textit{ True }. Tot în etapa de propagare a clauzelor unitare, se găsește acum clauza $c_5$, unde singurul literal rămas este $\neg x_6$, și deci variabila $x_6$ va primi valoarea \textit{False}. Astfel toate clauzele au valoarea \textit{True}, iar variabila $x_5$ poate avea orice valoare de adevăr. Deci s-a găsit o soluție validă și formula este satisfiabilă.

$x_1 = False ;~ x_2 = False ;~ x_3 = False ;~ x_4 =  True  ; ~x_5 = orice ;~ x_6 = False$

