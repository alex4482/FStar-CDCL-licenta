\chapter*{Concluzii} 
\addcontentsline{toc}{chapter}{Concluzii}


Prin urmare, desi acest 'solver' nu aplica multe tehnici de optimizare sau cele mai eficiente structuri de date, prezinta cum se poate folosi sistemul de verificare formala si demonstrare a limbajului FStar pentru algoritmul DPLL. Astfel toate specificatiile ce corespund verificarii algoritmului DPLL pot functiona ca o baza pentru specificatiile oricarei extensii al metodei DPLL, precum CDCL sau alte variante optimizate.

Solver-ul este dovedit a fi complet, corect, garantat ca produce un rezultat, iar generalizarea modulelor si interactiunii dintre ele poate insemna ca este un bun punct de plecare pentru crearea unui solver mai eficient folosind limbajul FStar, aplicant imbunatatiri precum cele prezentate pe scurt in aceasta lucrare. 


Implementarea descrisa pentru algoritmul DPLL se poate gasi urmaring link-ul: //LINK LA REPO
