\chapter{Detalii de implementare}


\section{Despre FStar}

Limbajul F* este un limbaj de programare orientat pe demonstratii ce poate fi folosit pentru uz general. Printre aspectele acestui limbaj se numara si faptul ca prezinta suport pentru programare functionala. De asemenea, este proiectat sa faca verificare formala a codului, astfel orice programator poate sa adauge functiilor specificatii in forma de pre-conditii, post-conditii, invarianti, pentru a putea asigura completitudinea codului.

\section{Completitudinea, corectitudinea si terminarea programului}

\subsection{Completitudinea programului}

Este dovedita completitudinea algoritmului DPLL pentru formula in forma normala conjunctiva (CNF), asa cum este el descris in Capitolul 1.

Singura optimizare a algoritmului DPLL care a fost implementata se leaga de verificarea existentei unei clauze false la fiecare pas recursiv dupa etapa de 'unit\_propagation' si asignarea unei valori unui literal necunoscut. Mai exact, la fiecare pas avem o lista de literali de lungime cel putin 1 care au fost adaugati la solutie, dar acesti literali noi nu schimba valoarea clauzelor care nu contin nici unul din literalii din lista.

Astfel aceasta optimizare se foloseste de o matrice de aparitie pentru fiecare literal din formula cate o lista cu toate clauzele in care apare respectivul literal. Se intersecteaza clauzele in care apar toti acesti literali asignati la pasul curent, iar cautarea unei clauze false se face doar peste aceasta multime, in loc de intreaga formula.

\subsection{Corectitudinea programului}

Pentru verificarea specificatiilor introduse de programator, se foloseste un SMT solver 

//ADAUGA REF LA SMT SOLVER. 

Cu toate acestea, unele conditii pot fi prea dificile de demonstrat intr-un punct anume al programului, unul din motivele pentru care limbajul F* ofera la dispozitia programatorilor  instructiuni ce simplifica misiunea de a demonstra si verifica proprietati si conditii scrise in program. De asemenea, in cadrul acestui proiect s-au folosit 'Lemme' pentru demonstrarea unor proprietati intre diferite tipuri de date ce respecta anumite conditii, si de asemenea instructiuni care ajuta 'SMT solver-ul' spunandu-i sa demonstreze intai proprietati intermediare.

//exemplu lemma si assert si la ce au ajutat a demonstra

Exemple asemanatoare se vor gasi si in sectiunile ce urmeaza.

Dupa verificarea unui fisier '.fst' se va extrage, in proiectul acesta in fisiere tipice limbajului Ocaml, codul ce este folosit la formarea rezultatului si este eliminat codul ce ajuta la specificatii si demonstratiile unor proprietati. Cu cat validarile devin mai complexe, cu atat se poate observa diferenta mare de linii de cod dintre fisierele '.fst' si cele '.ml'.

//exemplu raportul dintre nr lini

\subsection{Terminarea programului}

Cele mai importante proprietati pe care le respecta aceast 'SAT solver' sunt completitudinea algoritmului, corectitudinea implementarii si siguranta ca programul se va termina si va produce un rezultat.

Terminarea programului si producerea unui rezultat trebuie sa fie o certitudine, indiferent de valoarea rezultatului. Limbajul FStar poate verifica, daca este specificat acest lucru, daca o functie/un program se va termina si va produce un rezultat indiferent de parametrii primiti. De asemenea poate verifica daca programul/functia ar produce acelasi rezultat daca ar fi apelata de mai multe ori cu aceeiasi parametrii.

Programatorului i se ofera posibilitatea de a adnota functii folosind cuvantul "Tot" pentru a asigura ca la momentul compilarii se va incerca si demonstrarea faptului ca fiecare functie, mai ales cele recursive, se termina, produc un rezultat si mereu acelasi rezultat pentru aceleasi argumente. 

//exemplu 'Tot'

Toate metodele din modulele ce ajuta la formarea rezultatului algoritmului DPLL sunt 'functii totale', fie folosind adnotarea 'Tot' in signatura functiei, fie prin absenta oricarei adnotari, efectul 'Tot' fiind cel implicit. Succesul compilarii programului inseamna ca acesta se termina si fiecare input are asociat un singur rezultat. In cazul in care o formula are mai multe solutii, modul in care a fost implementat algoritmul va decide care din acele solutii se va returna.


\section{Structurile de date folosite}

Pentru acest proiect s-a ales proiectarea unor structuri de date si colectii simple, pentru care au fost create functii ce implementeaza operatii generice.

Conform definitiei, o formula este o multime de clauze, astfel implementarea are forma unei liste inlantuite de clauze.

// adauga linia cu font fain formula = list clause

O clauza, conform definitiei este o multime de literali, astfel este structurata ca o lista inlantuita de literali

//adauga linia cu font fain type clause = list literal

Un literal reprezinta o variabila a carui valoare de adevar poate fi negata, motiv pentru care tipul de data 'literal' este creat astfel, folosind 2 constructori 'Var' pentru a evidentia ca nu se neaga valoarea de adevar a variabilei, si 'NotVar' pentru a arata negatia variabilei.

//type literal = 
	| Var nat-non-zero
	| NotVar nat-non-zero
	
Doi literali, 'Var x' si 'NotVar x' sunt considerati a nu fi egali, insa au variabilele egale.
	
Variabilele sunt reprezentate folosind numere naturale nenule, deoarece un format des intalnit al datelor de intrare pentru aceasta problema si care s-a considerat la conceperea proiectului reprezentau literalii folosind numere intregi nenule. Am ales literalii sa fie reprezentati prin 2 constructori alaturi de numere intregi strict pozitive pentru a evita verificari repetate in cod cu privire la semnul unei variabile oarecare si pentru a se putea face usor distinctia intre tipurile de literali.

//type nat-non-zero = x : nat {x > 0}


Pentru matricea de aparitie a clauzelor si literalilor, s-a folosit o structura de date intermediara care reprezinta o linie a matricei si este formata dintr-o pereche de 2 elemente, un literal si o lista inlantuita de clauze:.

//type occurence-vector = {lit : literal ; clauses : list clause}

Conditia acestui tip de date este ca lista 'clauses' trebuie sa contina toate clauzele in care apare literalul 'lit' din formula primita la inceperea programului, invariant care a fost scris astfel:
//invariantul

Matricea in sine este o lista inlantuite de 'occurence-vector', si trebuie mereu sa respecte conditia ca nu exista 2 elemente in lista a caror membrul 'lit' sa fie egal ca si valoare.

//type occurence-matrix = list occurence-vector


Rezultatul functiei ce verifica satisfiabilitatea unei formule este in sine un o structura de date simpla creata pentru aceasta problema, reprezentata prin 2 constructori, unul pentru situatia in care formula e nesatisfiabila si unul care contine o posibila combinatie de variabile pozitive si valori booleene ce reprezinta o solutie valida pentru formula. 

//type result = Sat t | Unsat

In aceasta lucrare, se va referi la o combinatie de variabile pozitive si valori booleene prin termenul 'truth-assignment' prescurtat 't' sau 'tau'.

Acest tip de data 'truth\_assignment' reprezinta o lista inlantuita de elemente, structuri formate din doi parametri, unul 'value' ce reprezinta valoarea unei variabile si 'var' ce reprezinta in sine variabila.

//type variable-info = \{value : bool ; var : nat\_not\_zero\}

//type truth\_assignment = x : list variable\_info \{{truth\_assignment\_condition x\}


Acest 'truth\_assignment' trebuie sa respecte conditia ca nu exista 2 elemente in lista care sa aibe aceeasi valoare a membrului 'var'. Invariantul a fost scris astfel simplificand conditia, si anume ca membrul 'var' al elementului din capul listei trebuie sa nu apara in oricare alt element din lista, iar daca submultimea 'tail\_t' formata din toate elementele lui 't' exceptia capului listei trebuie sa respecte aceeasi proprietate.

//invariantul

Structurile de date au fost alese a fi simple in detrimentul eficientei vitezei de executie a programului, pentru a putea analiza dificultatea si complexitatea mentinerii corectitudinii in implementarrea unui algoritm complet in limbajul F*. 



\section{Modulele ce alcatuiesc programul}

Separarea programului pe mai multe fisiere s-a realizat pentru a usura dezvoltarea si modificarea codului in etapa implementarii. De asemenea, orice schimbare intr-un fisier rezulta in re-verificarea acestuia la momentul compilarii, proces care devine cu atat mai indelungat cu cat fisierul prezinta mai multe functii si demonstratii complexe.

Astfel modulele acestui program sunt:

DataTypes - fisier unde s-au definit toate structurile de date explicate anterior, invariantul pentru cele ce aveau nevoie de unul, si anumite functii ajutatoare pentru a manipula tipurile de date tip colectii demonstrand in acelasi timp corectitudinea operatiilor efectuate pe colectiile respective. 

Un exemplu de astfel de functie ar fi metoda
'add\_var\_to\_truth\_assignment', al carei simplu scop este de a adauga un nou element in colectia 'truth\_assignment', insa care trebuie sa poata respecta urmatoarea post\_conditie:

//exemplu post-conditie


Modulul 'DataTypeUtils' contine un numar considerabil de metode ajutatoare, folosite in mai multe alte module ale proiectului si care trebuie deci sa fie disponibile intr-un singur loc. Printre aceste metode apar:
\begin{itemize}
	\item  functii de procesesare a parametrilor primiti;
	
	ex: \textit{get\_clause\_value }- returneaza valoarea de adevar a unei clauze considerand un 'truth\_assignment' primit ca parametru;
	
	\item \textit{lemme} care ajuta la asigurarea si demonstrarea corectitudinii programului;
	
	ex: \textit{lemma\_no\_vars\_in\_t\_outside\_f\_length\_compare} folosita pentru a arata ca daca un 'truth\_assignment' nu contine nici o variabila care nu este prezenta in formula 'f', atunci sigur lungimea variabilei 't' este mai mica sau egala cu numarul variabilelor distincte ce apar in formula 'f';
	
	//exemplu
	
	\item pre/post-conditii importante si relevante demonstrarii corectitudinii salvate in variabile globale, pentru a reduce cantitatea de cod repetat si pentru a putea generaliza functiile la nivel inalt, avand posibilitatea in viitor de a optimiza aspecte ale programului precum structurile de date, pentru care ar trebui modificate doar implementarea metodelor ce proceseaza aceste structuri;
	
	ex: \textit{t\_cant\_be\_solution\_for\_f}, folosit ca post-conditie care enunta ca orice 'truth\_assignment' ce contine toate variabilele formulei 'f' si pentru care parametrul 't' este o sub-multime a sa, nu este o solutie valida pentru 'f';
	
	//exemplu
	
	
	\item predicate, functii care evalueaza daca anumiti parametrii respecta o anumita proprietate
	
	ex: \textit{is\_solution}, desi simplu, este unul din cei mai importanti predicati ai programului
	
	//exemplu
	
\end{itemize}


Modulul DpllPropagation conține puține metode însă importante pentru demonstrate si dificil la rândul lor de specificat si verificat. Aceste funcții se ocupă cu pasul de propagare a clauzelor 'unit' din algoritmul DPLL.

O mica parte însa cea mai importanta din specificarea metodei principale este:

//aseaza exemplul mai frumos

 t1\_is\_sublist\_of\_t2 t (fst res)
	/\ ((t\_cant\_be\_solution\_for\_f f t) <==> ( t\_cant\_be\_solution\_for\_f f (fst res)) )
	/\ length res.\_1 >= length t
	
Post-condițiile enunță următoarele: 
\begin{itemize}
	\item  faptul ca tot ce e inclus in 'tau' primit ca parametru trebuie sa fie existent si in 'tau' trimis ca rezultat; 
	
	\item lungimea 'tau'-ului rezultat trebuie sa fie macar egala cu lungimea 'tau'-ului primit, lucru care ajuta la asigurarea terminarii programului;
	
	\item 'tau'-ul primit ca parametru nu poate fi solutie pentru formula 'f' daca si numai daca 'tau'-ul rezultat nu poate fi solutie
\end{itemize}

Modulul 'OccurenceMatrix' contine functii necesare crearii, procesarii si accesarii matricei de aparitie a literalilor in clauzele formulei si de asemenea implementarea optimizata pentru metoda ce verifica daca un 'tau' oarecare este solutie partiala pentru formula data.

Modulul 'Dpll' contine functia principala ce primeste o formula si ofera un rezultat, unde se specifica si verifica legatura intre valoarea rezultatului si formula.

//exemplu cu post-conditia dpll

Modulul 'InputFileParser' a fost conceput pentru a putea folosi formule diverse de diferite dimensiuni prin parsarea unor fisiere de intrare ce respecta un anumit format.

//pune link in footer la site cu input files

 Modulul 'ConvertorToString' este folosit pentru a converti orice obiect creat de program intr-un sir de caractere si mai ales pentru a se afisa rezultatul pe ecran sub o forma usoara de citit si inteles.






\section{Metrici orientative}

Aceasta sectiune contine metrici aproximative sub forma unor timpi ce reprezinta durata de compilare/verificare a fisierelor sau timpul necesar obtinerii odata a rezultatului 'Satisfiabil' si odata a cazului in care formula nu este satisfiabila.


Algoritmul fiind unul simplu in comparatie cu optimizari moderne, timpul de demonstrare ca o formula este nesatisfiabila ajunge sa fie considerabil de mare chiar si pentru date de intrare cu dimensiuni relativ mici, motiv pentru care sunt necesare imbunatatiri ale algoritmului pentru a ajunge la o eficienta comparabila.

Compilatorul a avut nevoie de aproximativ 500 de secunde pentru verificarea fisierelor '.fst' si extragerea fisierelor '.ml' specifice limbajului Ocaml. 
\newline

\begin{tabular}{|l|l|} \hline
	Nume fisier & Durata verificare/compilare \\\hline
	DataTypes & 8 secunde \\\hline
	DataTypeUtils & 33 secunde \\\hline
	DpllPropagation & 285 secunde \\\hline
	OccurenceMatrixUtils & 91 secunde \\\hline
	Dpll & 71 secunde \\\hline
	ConvertorToString & 3 secunde \\\hline
	InputFileParser & 4 secunde \\\hline
	Main & 3 secunde \\\hline	
\end{tabular}
\newline\newline

Se observa faptul ca fisierele ce contin mai multe linii de cod pentru procesarea datelor si mai putine ce necesita demonstrarea unor proprietati simple, au nevoie de mai putin timp pentru a se verifica si complila. In schimb, fisierele care au foarte putine linii de cod ce contribuie la formarea rezultatului si ajung sa fie extrase in fisierele '.ml', dar multe proprietati mai complexe a caror demonstratie trebuie verificata, necesita mult mai mult timp de compilare.
\newline

Dispozitivul pe care s-a facut aceasta contorizare avea urmatoarele specificatii:
\begin{itemize}
	\item cpu - intel-core i5-1035G1
	\item memorie ram - 16gb
	\item tip hard-disk stocare - ssd
\end{itemize}



\section{Posibile optimizari}

Multi 'SAT solveri' moderi implementeaza algoritmul CDCL (Conflict-Driven-Clause-Learning) care, pe langa pasii din algoritmul DPLL, adauga un pas nou de invatare si eficientizeaza saltul inapoi in momentul in care se gaseste o solutie falsa.

//ADAUGA REFERINTA LA CDCL

Un alt rafinament pe langa imbunatatirea algoritmului implementat poate fi eficientizarea structurilor de date folosite, pentru a reduce complexitatea timp a operatiilor ce se repeta cel mai des.

O problema ce poate aparea si in implementarea algoritmul DPLL dar si cea a algoritmului mai complex CDCL, este o metoda/euristica ineficienta de a alege o noua variabila pentru a i se asigna o valoare, si insasi valoarea care sa i se asocieze. Un exemplu ar fi metoda VSIDS, care se bazeaza pe asocierea unui scor pentru fiecare variabila, scor care se schimba daca variabila apare intr-un conflict. 

//ADAUGA REFERINTA PT MAI MULTE INFORMATII LA VSIDS SAU ALTE EURISTICI

O alta optimizare se poate implementa prin paralelizarea calculelor si impartirea ramurilor/solutiilor ce trebuie explorate.

O mica imbunatatire din perspectiva limbajului de programare de aceasta data poate fi detectarea instructiunilor ce au rolul de a ajuta demonstrarea unor proprietati insa ajung pe parcursul dezvoltarii programului sa devina redundante, iar eliminarea lor ar putea reduce putin timpii de validare/compilare.

De asemenea, mai exista varianta de a simplifica problema pe care algoritmul trebuie sa o rezolve, iar analizand fiecare clauza in parte poate duce la identificarea unor clauze si literali redundanti, care pot fi eliminati din formula primita fara a schimba completitudinea si corectitudinea rezultatului final.