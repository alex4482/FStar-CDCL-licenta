\chapter*{Introducere} 
\addcontentsline{toc}{chapter}{Introducere}

Problema satisfiabilitatii booleene (SAT) se regaseste in multe ramuri ale \linebreak domeniului informaticii, precum in verificarea de software si hardware, in domeniul criptografiei si securitatii sau optimizarea si analiza altor programe.

'SAT' este considerata prima problema ce a fost demonstrata a fi NP-completa \footnote{https://en.wikipedia.org/wiki/Cook\%E2\%80\%93Levin\_theorem}, motiv pentru care este importanta dezvoltarea unor programe care sa poata gasi o solutie cat mai rapid, eficient si sigur. 

In ziua de astazi exista si inca se dezvolta si optimizeaza 'SAT solveri' eficienti, cum ar fi cei care au participat la concursul "SAT 2022": \href{https://www.dominikschreiber.de/papers/2022-mallob.pdf}{Mallob}, sau \href{https://github.com/maximaximal/Paracooba#readme
}{Paracooba}.

Algoritmul DPLL este una din primele optimizari eficiente folosite in rezolvarea problemei SAT. Este considerat a fi un punct de plecare fundamental pentru \linebreak dezvoltarea unui 'SAT solver' eficient, stand la baza multor 'SAT solveri' moderni.

Insa chiar si o implementare a unui algoritm demonstrat a fi eficient si complet poate produce erori sau rezultate gresite, iar unul din motivele pentru care apar aceste erori este eroarea umana a programatorului, a carei frecventa poate creste impreuna cu complexitatea algoritmului ce trebuie implementat. Pentru ca este testarea unui 'SAT solver' nu poate garanta absenta erorilor, s-au conceput programe care sa verifice ca implementarea in sine a algoritmului este lipsita de erori si matematic corecta.

Limbajul de programare F* (FStar) pune la dispozitie un astfel de sistem care in momentul compilarii executa o verificare formala a programului in functie de \linebreak specificatiile introduse de programator. Acest sistem ajuta la depistarea bug-urilor, prevenirea erorilor, si ofera o siguranta mai mare ca programul este corect \linebreak implementat si nu va prezenta rezultate false sau gresite.

Astfel, am ales sa implementez algoritmul DPLL folosind limbajul de programare FStar, pentru a evidentia cum arata parcursul crearii si verificarii unui 'SAT solver' in acest limbaj si cum se poate extinde din punct de vedere al eficientizarii.

Aceasta lucrare va prezenta in continuare o implementare a algoritmului DPLL folosind limbajul FStar. Vor fi prezentate detalii despre algoritmul DPLL, despre limbajul de programare folosit, detalii de implementare cu exemple de cod si motivele pentru care 'SAT solver-ul' rezultat este complet, corect si produce mereu un rezultat.


