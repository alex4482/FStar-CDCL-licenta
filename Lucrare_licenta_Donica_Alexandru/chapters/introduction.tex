\chapter*{Introducere} 
\addcontentsline{toc}{chapter}{Introducere}

Problema satisfiabilității booleene (SAT) se regăsește în multe ramuri ale \linebreak domeniului informaticii, precum în verificarea de software și hardware, in domeniul criptografiei și securității sau optimizarea și analiza altor programe.

'SAT' este considerată prima problema ce a fost demonstrată a fi \linebreak NP-completă \footnote{https://en.wikipedia.org/wiki/Cook\%E2\%80\%93Levin\_theorem}, motiv pentru care este importantă dezvoltarea unor programe care să poată găsi o soluție cât mai rapid, eficient și sigur. 

În ziua de astăzi exista și încă se dezvoltă și optimizează 'SAT solveri' eficienți, cum ar fi cei care au participat la concursul "SAT 2022": \href{https://www.dominikschreiber.de/papers/2022-mallob.pdf}{Mallob}, sau \href{https://github.com/maximaximal/Paracooba#readme
}{Paracooba}.

Algoritmul DPLL reprezintă una din primele optimizări eficiente folosite în \linebreak rezolvarea problemei SAT. Este considerat a fi un punct de plecare fundamental pentru \linebreak dezvoltarea unui 'SAT solver' eficient, stând la baza multor solveri moderni.

Însă chiar și o implementare a unui algoritm demonstrat a fi eficient și că acoperă fiecare caz posibil poate produce erori sau rezultate greșite, iar unul din motivele pentru care apar aceste erori este eroarea umana a programatorului, a cărei frecvență poate crește împreună cu complexitatea algoritmului ce trebuie implementat. Pentru că testarea unui 'SAT solver' nu poate garanta absența erorilor, s-au conceput programe care sa verifice că implementarea în sine a algoritmului este lipsită de erori și matematic corectă.

Limbajul de programare F* (FStar) pune la dispoziție un astfel de sistem care în momentul compilării execută o verificare formală a programului în funcție de \linebreak specificațiile introduse de programator. Acest sistem ajută la depistarea bug-urilor, prevenirea erorilor, ăi oferă o siguranță mai mare că programul este corect \linebreak implementat și nu va prezenta rezultate false sau greșite.

Astfel, am ales sa implementez algoritmul DPLL folosind limbajul de programare FStar, pentru a evidenția cum arată parcursul creării și verificării unui 'SAT solver' în acest limbaj și cum se poate extinde din punct de vedere al eficientizării.

Această lucrare va prezenta în continuare o implementare a algoritmului DPLL folosind limbajul FStar. Vor fi prezentate detalii despre algoritmul DPLL, despre limbajul de programare folosit, detalii de implementare cu exemple de cod și motivele pentru care 'SAT solver-ul' obținut este corect și produce mereu un rezultat.


