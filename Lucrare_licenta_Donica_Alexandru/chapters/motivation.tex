\chapter*{Motivație} 
\addcontentsline{toc}{chapter}{Motivație}

Rolul unui 'SAT solver' este de a rezolva problema satisfiabilității booleene, \linebreak făcându-se si in ziua de azi cercetări care au scopul de a îmbunătăți algoritmi existenți. Acest 'SAT solver' găsește o soluție pentru o formula data in cazul in care formula este satisfiabilă, in caz contrar formula este nesatisfiabilă. 

Corectitudinea oricărui rezultat al unei formula satisfiabile poate fi verificata
\linebreak folosind algoritmi simpli. Însă un 'SAT solver' complex creat pentru a procesa \linebreak formule de dimensiuni din ce in ce mai mari sau pentru a avea o viteza de rezolvare a \linebreak problemei mai rapida decât alți 'SAT solveri', poate conține erori de programare ce ar produce rezultate false, cum ar fi, in urma procesării unei formula nesatisfiabile acesta sa enunțe ca este satisfiabilă, sau invers.

De aceea este importanta crearea unor programe care verifica corectitudinea \linebreak acestor 'SAT solveri' si am creat un 'SAT solver' verificat formal folosind limbajul de programare F* (FStar).