\chapter*{Motivație} 
\addcontentsline{toc}{chapter}{Motivație}

Rolul unui 'SAT solver' este de a rezolva problema satisfiabilității booleene, \linebreak făcându-se și in ziua de azi cercetări care au scopul de a îmbunătăți algoritmi existenți. Acest 'SAT solver' găsește o soluție pentru o formulă dată in cazul în care formula este satisfiabilă, in caz contrar formula este nesatisfiabilă. 

Corectitudinea oricărui rezultat al unei formule satisfiabile poate fi verificată
\linebreak folosind metode simple. Însă un 'SAT solver' complex creat pentru a procesa \linebreak formule de dimensiuni din ce in ce mai mari sau pentru a avea o viteză de rezolvare a \linebreak problemei mai rapidă decât alți 'SAT solveri', poate conține erori de programare ce ar produce rezultate false, cum ar fi, in urma procesării unei formule nesatisfiabile acesta să enunțe că este satisfiabilă, sau invers.

De aceea este importantă crearea unor programe care verifică corectitudinea \linebreak acestor 'SAT solveri' și am creat un 'SAT solver' verificat formal folosind limbajul de programare F* (FStar).