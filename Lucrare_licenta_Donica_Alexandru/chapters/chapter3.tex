\chapter{Pasi necesari pentru a reproduce}

Informatii despre replicarea conditiilor necesare executiei programului scris folosind F* se pot gasi pe pagina de github a limbajului F*. \footnote{https://github.com/FStarLang/FStar/blob/master/INSTALL.md}

In sectiunea urmatoare se prezinta pasii care au fost luati pentru crearea \newline mediului in care s-a dezvoltat 'SAT solver-ul'. Instructiunile urmatoare sunt \newline compatibile cu sistemul de operare Windows, verificat cu versiunile 10/11. 

\section{Programe si resurse necesare}

Urmatoarele aplicatii/programe/resurse trebuie descarcate de la locatia specificata fiecareia in fisierul 'INSTALL.md' gasit pe pagina de github a limbajului FStar.

\begin{itemize}
 \item OCaml - necesar compilarii si executarii fisierelor OCaml (.ml), care rezulta in urma compilarii fisierelor FStar (.fst)
	(versiune folosita - 4.14.0)
 \item opam - folosit pentru a instala pachetele necesare compilarii fisierelor specifice limbajului de programare OCaml 
 
 (versiunea folosita pentru lucrare - 2.0.10)
 \item cygwin - ofera posibilitatea compilarii si executarii a programelor tipice \newline sistemelor de operare Unix si Linux, ceea ce include suport pentru fisiere \newline  'Makefile' 
 (versiunea folosita pentru lucrare - 3.4.6)
 \item Z3 - folosit pentru a valida fisierele ce contin programe scrise folosind F*
 

 
 (arhiva folosita pentru Windows - z3-4.8.5-x64-win.zip)
\end{itemize}

\newpage


Dupa descarcarea/instalarea acestor resurse, trebuie clonata ramura 'master' a proiectului FStar pe dispozitivul local. (locatia clonei pe dispozitivul folosit pentru acest proiect: "D:/fstar", versiunea  - F* 2023.04.26~dev )

Trebuie adaugate path-urile absolute catre ".../fstar/bin" si ".../z3-win/bin" in variabila 'Path' a sistemului.

Dupa instalarea programului 'opam', trebuie instalate mai anumite pachete de date. Minimul necesar de pachete se poate gasi si pe instructiunile de instalare gasite la link-ul de mai sus, insa pentru a avea la dispozitie toate resursele din proiectul FStar descarcat fara erori, sunt necesare urmatoarele pachete: 
\newline

\includegraphics[width=1\textwidth]{opam_necesary_packages.png}
\newline

La finalul acestor pasi, folosind terminalul Cygwin si instructiunile de tipul 'make' in folder-ul 'fstar', ar trebui sa functioneze verificarea si executarea oricaror fisiere surse scrise in F*, fisiere proprii sau exemple ce faceau deja parte din proiect.

\newpage

\section{Executarea solver-ului SAT}

Sursele corespunzatoare proiectului prezentat se gasesc la:
\href{https://github.com/alex4482/FStar-DPLL-licenta/tree/main/dpll_optimized}{FStar-DPLL github}.

Aceste surse trebuie descarcate, salvate intr-un folder in proiectul 'fstar'. \newline Fisierul 'Makefile' trebuie modificat, astfel incat variabila 'FSTAR-HOME'  sa faca \newline referire folder-ul 'fstar'. Acelasi pas trebuie facut pentru fisieru 'Makefile' din \newline folder-ul 'output'.

Apoi, in terminalul cygwin deschis in folder-ul proiectului DPLL-FStar, trebuie executata comanda 'make', la finalul careia in folder-ul 'output' vor aparea pentru \newline fiecare fisier sursa '.fst' cate un fisier '.ml' care contin codul Ocaml extras din sursele FStar. De asemenea in folder-ul 'output' se va afla executabilul "Main.exe".

Pentru a recompila si regenera fisierul "Main.exe", trebuie sters cel anterior creat, daca a fost creat.

Imediat dupa pornirea programului "Main.exe", trebuie introdus de la tastatura calea relativa catre un fisier de input. Cateva fisiere de input exista in folder-ul \newline "input-files" si orice alt fisier de intrare trebuie sa respecte acea structura pentru ca parsarea datelor implementata sa functioneze.

La finalul unei astfel de executii, va aparea mesaj la consola cu rezultatul obtinut, fie ca formula data este nesatisfiabila, fie ca e satisfiabila si alaturi o varianta de raspuns ce contine variabilele formulei si valorile lor booleene astfel incat fiecare clauza a formulei sa aibe valoarea de adevar true.
\newline

\includegraphics[width=0.8\textwidth]{unsat-sat-result-execute-example.png}

%\newline
%\includegraphics[width=0.3\textwidth]{unsat-result-execute-example.png}
%\newline
